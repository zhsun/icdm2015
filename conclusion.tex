\section{Conclusions}
\label{sec:conclusion}
In summary, the Similarity Powered Pairwise Amplifier Network
({\sppan}) model we proposed in this article shows promising results
in predicting click-through rate (CTR) in a very sparse CTR data
set. It utilizes the pairwise ratio similarity information that
embedded in the data set, and its model complexity is automatically
adjusted according to the sparsity of the data set. Comparing with
several matrix factorization based recommendation methods, our
evaluation experiment results show that this model can substantially
improving the performance of the recommendation system in extreme
sparse situations. It has also been shown that the training process of
the {\sppan} model can be easily implemented in a paralleled version
through map-reduce, which makes the model can handle even bigger data
set efficiently.

Since all evaluation experiments in this paper were performed on a
click-through rate data set, we must observe that more experimental
evaluations of the proposed {\sppan} model should be performed in the
future in order to make a more comprehensive conclusion of the
performance of the model. In fact, out next plan is to apply the
{\sppan} model on other sparse data sets from different domains so
that we could confirm if the {\it Homogeneous Amplifying Effect}
described in Section \ref{sec:model_intuition} is valid only in online
advertisement traffic data or can be generalized in other
areas. Moreover, the experiment result in the present paper only shows
the advantages of the {\sppan} model in a situation with a certain
sparsity level. A sensitivity analysis about the effect of the
sparsity on the model's performance would help us to decide when to
use {\sppan} model instead of other approaches.
