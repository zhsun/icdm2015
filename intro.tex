\section{Introduction}
\label{sec:intro}

Personalized recommendation is widely applied to various online
systems, including online advertising, e-commerce platforms, social
networks, etc. Online advertising plays an important role in majority
of e-commerce sites. It is a multi-billion dollar business, and brings
tremendous revenue to companies like Google and Facebook. Take Google
as an example, its online adversiting system generates a large portion
of Google's revenue, supports hundreds of millions of publishers on
the Internet eco-system, and can reach more than 80\% of all Internet
users worldwide in more than 30 languages and over 100 countries.

%All this makes it one of advertisers' favorite online advertising products.

To make the best out of the online advertising system, it requires the
advertiser to provide a good candidate keywords list, to match
advertisements with web contents more precisely, hence increase the
click through rate, and increase the product revenue by
advertisements. As most advertisers may be lack of experience to
provide good keywords for their advertisements, especially when they
start running advertisement for the first time, historical
advertisement traffic data is used by recommendation system to offer
advertisers personalized keyword suggestions. The suggested keywords
need to be relevant to the products they are promoting, as well as
bring a lot of traffic (e.g. user views, clicks, etc.) to the
advertisement.

The process of keyword recommendation using historical data is usually
composed of two steps: first, a candidate list of semantic relevant
keywords for each advertisement is generated. For instance, if the
advertisement is about a new digital camera, ``electronics'' and
``photography'' would be semantically relevant keywords. Secondly,
based on candidate keyword list, a further keyword refinement and
selection is executed to generate the final keyword lists.

There are mainly two types of methods in the second step.  The most
popular one is A/B testing~\cite{abtest:wiki}, which split the traffic
into two parts: one big part still feeds into the original keywords
and the other small part feeds into the newly suggested keywords, and
compare the difference of these two parts, to decide how much more
traffic can be brought by adding the new keywords, hence update the
new lists with the most beneficial keywords. A/B testing usually takes
reasonably long period of time (e.g. several weeks) to decide which
keyword should be suggested. During that long turnover time, lots of
business opportunities may be lost. Moreover, A/B testing only uses a
small portion of the real traffic information, the data may be biased
and not error-prone, hence generate inaccurate estimation.

Another alternative solution for step two, is to use model based
methods, e.g. Collaborative Filtering~\cite{resnick1997recommender,
  sarwar2001item}, which makes automatic predictions (filtering) about
the preference of one advertisement by collecting preferences from
many others (collaborating). In practice, the recommendation system
are based on large datasets, as a result, user based or item based
matrix used for collaborative filtering could be extremely large and
sparse, which challenges the model performance. Matrix factorization
models \cite{??} which can better handle the data sparsity compared
with item and user based method.  However, we have noticed that due to
the nature of extremely sparse data set in advertisement, the matrix
factorization based models tend to over-fit in the training data set
even with very strong regularization. As a result, the performance of
matrix factorization based estimation models is not suitable for the
test data set.

We consider a machine learning-based approach to building an effective
recommendation system. Such an approach should process the large scale
data in a scalable manner to make good recommendations, and provide an
cost-effective list of most valuable keywords to end users. It should
provide the recommendation and keyword selections in a time sensitive
manner, with a very fast turn around time, without losing any business
opportunity.  It should support very sparse data, with more accurate
predictions. It should handle both training and testing data set
properly to avoid the over-fitting.
 
We proposed a novel approach that addresses the above
requirements. The motivation for our model comes from the similar idea
as collaborative filtering, that similarity ratio of different items
between similar users should be close to each other. It gains insights
from both item-based similarity and user-based similarity. Instead of
using only item-based or user-based method, it is a synthetic
multi-dimension model, which evaluates the similarity cross item and
user dimensions. Also, different from item-based or user-based method,
which uses a predefine static similarity function and calculate the
global optimal similarity, we are using a local optimal function,
which can dynamic involving similarity using learned parameters, hence
can support large scale data in an efficient computation cost.  We
have developed an innovative non-parametric model: Similarity powered
Pairwise Amplifier Network (SPAN) model. It uses observed performance
ratios between keyword pairs to give prediction for the unknown
entries. Comparing to matrix factorization based methods like
LSNMF[complete name + referecen??] and NMF[complete name +
  reference?], experiments on the same Click-through rate data set
show that the proposed SPPAN model increases the prediction accuracy
by more than $50\%$.

In summary, our contributions in this paper include:
\begin{itemize} \itemsep -2pt
\item We successfully exploit the potential of improving
  recommendation for sparse and large scale data.
\item We develop a … model, which can adjust the complexity of the
  model automatically according to the sparsity of the dataset, which
  makes it more accorded to extreme sparse data set.
\item We conduct extensive experiments to verify out approach and
  compare with existing methods, and demonstrate that SPANN can
  effectively resolve the data sparsity and scalability. Thus, it can
  be adopted as a recommendation tool for general propose.
\end{itemize}

\textcolor{blue}{ 1. use ratio, 2. parameter propostion to the
  observed data; factorization propotion to user and item main
  contribution: (1)one set of problem, extreme sparse data (2)model
  complexity propotion to the observed data (3)app to adsense (4)
  implementation distribution framework } The remainder of this paper
is organized as follows. We first formulate the problem in Section
\ref{sec:problem} Then, we describes the details of SPPAN model,
including the high-level work-flow of the model, the intuition and
assumptions behind the proposed SPPAN model, and different components
in the model. It also established the math expressions and notations
that are used throughout the rest of the paper in Section
\ref{sec:model} .After that, we introduce the training algorithms of
SPPAN model in Section \ref{sec:trainer}, which includes the Feed
Forward Estimation process and the Error Back Propagation process. In
Section \ref{sec:exp}, we evaluate the SPPAN model through prediction
experiments on a Click-through rate data set. Experiment results show
that the SPPAN model performs much better than matrix factorization
based NMF and LSNMF methods on predicting unknown values on this super
sparse data set. Section \ref{sec:related} shows several current
approaches and researches that try to solve the similar
problem. Finally, we concluded the whole article in Section
\ref{sec:conclusion}, where we highlighted the key findings and
achievements of this paper.
