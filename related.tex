\section{Related Work}
\label{sec:related}

Among the prevalence in online advertisements system, e-commerce
platform and social networks, there has been an increasing interest in
developing recommendation systems \cite{resnick1997recommender,
  ricci2011recommender} that can automatically predict the
"preference", "rating", "performance" or "interest" of a item for a
end user or a client based the history data, such as the task we
introduces in Section \ref{sec:intro}. One popular technique that has
been used in many recommendation systems is called collaborative
filtering, which is based on the intuition that if two users(or
clients) have a similar opinion on an item(or issue), they are more
likely to have similar opinions on other items(or
issues)\cite{resnick1997recommender, sarwar2001item,
  koren2011advances, ricci2011recommender}. Generally, collaborative
filtering is an approach that make use of the preference information
from many users to give prediction of the interest of one user. There
are many forms of collaborative filtering systems. For examples,
Linden et al. uses the item-to-item collaborative filtering to offer
personalized recommendations for each customer in the online
store\cite{linden2003amazon}; Cai et al. captures the interaction
between users within a social network and formulates a collaborative
filtering approach to allow high quality people to people
recommendations in social networks \cite{cai2011collaborative}; Hu et
al. implements a large scale TV recommender system with a
collaborative filtering based on prior implicit feedback that can
recommend new TV programs to their users with high accuracy
\cite{hu:2008}.

As a special type of collaborative filtering, various forms of matrix
factorization based recommendation system were used by researchers for
different recommendation tasks \cite{ZitnikZ12, lin2007projected,
  lee2001algorithms, brunet2004metagenes, parambath2013matrix,
  ricci2011recommender}. For instance, to help Flickr users more
easily engage in group activities, Zhang et al. proposes a tensor
decomposition-based Flickr group recommendation model, which is based
on CANDECOMP/PARAFAC tensor decomposition method to capture the
underlying patterns in the user-tag-group
relations\cite{zheng2010flickr}. Gu et al. introcudes a graph
regularized nonnegative matrix factorization model for general
collaborative filtering tasks, which outperform many state of the art
collaborative filtering methods on benchmark data
sets\cite{gu2010collaborative}. Other work by Baltrunas et
al. presents an context-aware matrix factorization (CAMF) method which
extends the classical Matrix Factorization approach by taking
contextual information into consideration
\cite{baltrunas2011matrix}. By applying their method on MovieAT data
set and Yahoo Webscope movie data to do movie rating prediction,
Baltrunas et al . have shown that the CAMF method can substantially
improve the rating prediction accuracy comparing to the the classical
Matrix Factorization approaches in certain circumstances in which the
relevant context information is available.

Despite there being a lot of researches of matrix factorization based
collaborative filtering methods, many algorithms still have salient
weaknesses under sparsity conditions
\cite{cacheda2011comparison}. Even though there are approaches like
multi-Domain collaborative filtering by Zhang et al. and adapting
neighborhood and matrix factorization model by Liu et al. trying to
improve the model's performance by integrating other available context
information\cite{zhang2012multi, liu2010adapting}, an extreme sparse
data set would easily make most of the matrix factorization based
models over-fitting. The {\sppan} model in this present paper, however,
adjust the complexity of the model automatically according to the
sparsity of the data set, which makes it more accustomed to extreme
sparse data set.
